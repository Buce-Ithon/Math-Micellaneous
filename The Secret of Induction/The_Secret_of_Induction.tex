\documentclass{Math_Note}

\title{The Secret of Induction \\ \small ------A Courese Note of MIT6.042 E2}
\author{Buce-Ithon}
\newdateformat{mydate}{\twodigit{\THEDAY}{ }\shortmonthname[\THEMONTH], \THEYEAR}
\date{\today}

\begin{document}

%Title page
\maketitle

%Content
\newpage
\tableofcontents
\newpage

%Beginning
\section{Introduction}
Mathmatical induction is a common method, or exactly an important idea, in numerous mathematical proofs. However, how much do we really know about it?
How deeply do we understand it? 

In this article, I will try to explain the secret of induction, and show you how to use it in a more flexible way with 
deeper comprehension.

\section{Two Simple Examples}
Before we starting, let's describe the basic concept of induction. Induction is a method to prove a statement for all positive integers or to say situations: 

\begin{mdframed}
\begin{thm}[Induction Axiom]
    Let $P(n)$ be a statement or predicate involving a positive integer $n$. If 
    \begin{enumerate}
        \item $P(1)$ is true, and
        \item For all positive integers $k$, if $P(k)$ is true, then $P(k+1)$ is true,
    \end{enumerate}
    then $P(n)$ is true for all positive integers $n$.
\end{thm}
\end{mdframed}

Now let's see two simple examples to show the basic idea of induction.

\begin{prb}
    Prove that $\forall n \in \mathbb{N}$, the sum of the first $n$ positive integers $\Sigma_{k=1}^{n} k=\frac{n(n+1)}{2}$.
\end{prb}

\begin{sol}
    Let $P(n)$ be the statement/predicate that the sum of the first $n$ positive integers: $\Sigma_{k=1}^{n} k=\frac{n(n+1)}{2}$. We will prove $P(n)$ is true for all positive integers $n$.

    \textbf{Base Case:} $P(1)$ is true, since the sum of the first positive integer is $1 = \frac{1(1+1)}{2}$.

    \textbf{Inductive Step:} Assume that $P(k)$ is true for some positive integer $k$, that is, the sum of the first $k$ positive integers is $\frac{k(k+1)}{2}$. We need to show that $P(k+1)$ is true, that is, the sum of the first $k+1$ positive integers is $\frac{(k+1)(k+2)}{2}$.

    By the inductive hypothesis, the sum of the first $k+1$ positive integers is 
    \begin{align*}
        1+2+\cdots+k+(k+1) &= \frac{k(k+1)}{2} + (k+1)\\
        &= \frac{k(k+1)+2(k+1)}{2}\\
        &= \frac{(k+1)(k+2)}{2}.
    \end{align*}
    Thus, $P(k+1)$ is true.

    By the inductive axiom, $P(n)$ is true $\forall n \in \mathbb{N}$.
\end{sol}

The next problem is a little bit more complicated, but we can handle it with this axiom.

\begin{prb}
    Prove that $\forall n \in \mathbb{N}$, $3 \mid n^{3}-n$.
\end{prb}

\begin{sol}
    Let $P(n)$ be the statement/predicate that $3 \mid n^{3}-n$. We will prove $P(n)$ is true for all positive integers $n$.

    \textbf{Base Case:} $P(1)$ is true, obviously $1^{3}-1=0$ is divisible by 3.

    \textbf{Inductive Step:} Assume that $P(k)$ is true for some positive integer $k$, i.e. $3 \mid k^{3}-k$. We need to show that $P(k+1)$ is true, i.e. $3 \mid (k+1)^{3}-(k+1)$.

    By the inductive hypothesis, $3 \mid k^{3}-k$, that is, $k^{3}-k=3m$ for some integer $m$. Then we have: 

    \begin{align*}
        (k+1)^{3}-(k+1) &= k^{3}+3k^{2}+3k+1-k-1\\
        &= k^{3}-k+3k^{2}+3k\\
        &= 3m+3k^{2}+3k\\
        &= 3(m+k^{2}+k).
    \end{align*}
    Thus, $3 \mid (k+1)^{3}-(k+1)$.

    By the inductive axiom, $P(n)$ is true $\forall n \in \mathbb{N}$.
\end{sol}

According to the two examples above, we can make a small conclusion for the induction axiom:

\begin{enumerate}
    \item \textbf{Step1.} State our predicate $P(n)$.
    \item \textbf{Step2.} Prove the base case $P(1)$ is true.
    \item \textbf{Step3.} Assume that $P(k)$ is true and prove that $P(k+1)$ is true, $\forall k \in \mathbb{N}$.
    \item \textbf{Step4.} Give conclusion: by the induction axiom, $P(n)$ is true $\forall n \in \mathbb{N}$.
\end{enumerate}

Besides, as you can see, there are some shortcuts in the induction proof. 

\begin{enumerate}
    \item Induction cann't help us find the answers of problems, it only helps us to prove the correctness of the answers.
    \item Induction is not good for us to understanding the problems, we cann't get more details or insight of the problems by induction.
\end{enumerate}

\section{A Seemly Wired Example}

All right, we have seen some introduction sample questions which are easy to help us being familiar with induction. Now, let's see a "seemly" 
wired problem with a wrong induciton solution to help us understand depper about the axiom.

\begin{prb}
    Prove that all horses are the same color.
\end{prb}

\begin{sol}
    Let $P(n)$ be the statement/predicate that: all horses in a group of $n$ horses are the same color. We will prove $P(n)$ is true for all positive integers $n$.

    \textbf{Base Case:} $P(1)$ is true, since there is only one horse in the group, it is the same color as itself.

    \textbf{Inductive Step:} Assume that $P(k)$ is true for some positive integer $k$, that is, all horses in a group of $k$ horses are the same color. We need to show that $P(k+1)$: all horses in a group of $k+1$ horses are the same color is true.

    By the inductive hypothesis, all horses in a group of $k$ horses are the same color. Let's mark it that: $H_{1}=H_{2}=\cdots=H_{k}$.
    
    Now, we add $1$ more horse to the group, then we have a group of $k+1$. Since all horses in a group of $k$ horses are the same color,

    then we have $H_{1}=H_{2}=\cdots=H_{k}$ and $H_{2}=H_{3}=\cdots=H_{k+1}$, then $H_{1}=H_{k+1}$, which means all horses in a group of $k+1$ horses are the same color.

    By the inductive axiom, $P(n)$ is true $\forall n \in \mathbb{N}$.
\end{sol}

Unfortunately, the proof is wrong, not just because the problem itself is wrong even we use induction to get a true result, but the most important is that our usage of induction makes some mistakes.

\textbf{Analysis:}

The mistake appears in the inductive step: pay your attention to the situation when $k=1$. In this case, we have only 2 horses in the group, and the inductive hypothesis is that all horses in a group of $1$ horse are the same color. 
This means we now only have $H_{1}=H_{1}$ and $H_{2}=H_{2}$, then if we make the next step as $H_{1}=H_{2}$, that may be not true. This the key point of the mistake what we made.

The wired point of this problem is that we know it is a wrong fact, but we get a true result by using correct induction(not correctly using induction:O).

\textbf{And what can we learn from that?} 

It tells us that when we use induction method to proof a statement, we need to make sure that the inductive step is correct, and even we need to check the base case carefully.

\section{Conclusion}

After the above statements and analysis, we can make a conclusion for the induction axiom:

\begin{enumerate}
    \item Induction Axiom
    \begin{enumerate}
        \item \textbf{Step1.} State our predicate $P(n)$.
        \item \textbf{Step2.} Prove the base case $P(1)$ is true.
        \item \textbf{Step3.} Assume that $P(k)$ is true and prove that $P(k+1)$ is true, $\forall k \in \mathbb{N}$.
        \item \textbf{Step4.} Give conclusion: by the induction axiom, $P(n)$ is true $\forall n \in \mathbb{N}$.
    \end{enumerate}
    \item Some thoughts:
    \item [shortcuts]Induction always cann't help us find the answers of problems(sometimes maybe), it only helps us to show the correctness of the statement.
    \item [attentions]Induction axiom is true, but in proof process we need to make sure that the inductive step is correct(especially some particular steps). Moreover, we even need to check the base case carefully.
\end{enumerate}

Finally, I hope this article can help you to understand the induction axiom more deeply and use it more flexibly:).

\end{document}